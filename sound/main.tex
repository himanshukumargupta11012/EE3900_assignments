\documentclass[journal,12pt,twocolumn]{IEEEtran}
%
\usepackage{setspace}
\usepackage{gensymb}
\usepackage{xcolor}
\usepackage{caption}
%\usepackage{subcaption}
%\doublespacing
\singlespacing

%\usepackage{graphicx}
%\usepackage{amssymb}
%\usepackage{relsize}
\usepackage[cmex10]{amsmath}
\usepackage{mathtools}
%\usepackage{amsthm}
%\interdisplaylinepenalty=2500
%\savesymbol{iint}
%\usepackage{txfonts}
%\restoresymbol{TXF}{iint}
%\usepackage{wasysym}
\usepackage{hyperref}
\usepackage{amsthm}
\usepackage{mathrsfs}
\usepackage{txfonts}
\usepackage{stfloats}
\usepackage{cite}
\usepackage{cases}
\usepackage{subfig}
%\usepackage{xtab}
\usepackage{longtable}
\usepackage{multirow}
%\usepackage{algorithm}
%\usepackage{algpseudocode}
%\usepackage{enumerate}
\usepackage{enumitem}
\usepackage{mathtools}
%\usepackage{iithtlc}
%\usepackage[framemethod=tikz]{mdframed}
\usepackage{listings}


%\usepackage{stmaryrd}


%\usepackage{wasysym}
%\newcounter{MYtempeqncnt}
\DeclareMathOperator*{\Res}{Res}
%\renewcommand{\baselinestretch}{2}
\renewcommand\thesection{\arabic{section}}
\renewcommand\thesubsection{\thesection.\arabic{subsection}}
\renewcommand\thesubsubsection{\thesubsection.\arabic{subsubsection}}

\renewcommand\thesectiondis{\arabic{section}}
\renewcommand\thesubsectiondis{\thesectiondis.\arabic{subsection}}
\renewcommand\thesubsubsectiondis{\thesubsectiondis.\arabic{subsubsection}}

%\renewcommand{\labelenumi}{\textbf{\theenumi}}
%\renewcommand{\theenumi}{P.\arabic{enumi}}

% correct bad hyphenation here
\hyphenation{op-tical net-works semi-conduc-tor}

\lstset{
language=Python,
frame=single, 
breaklines=true,
columns=fullflexible
}



\begin{document}
%

\theoremstyle{definition}
\newtheorem{theorem}{Theorem}[section]
\newtheorem{problem}{Problem}
\newtheorem{proposition}{Proposition}[section]
\newtheorem{lemma}{Lemma}[section]
\newtheorem{corollary}[theorem]{Corollary}
\newtheorem{example}{Example}[section]
\newtheorem{definition}{Definition}[section]
%\newtheorem{algorithm}{Algorithm}[section]
%\newtheorem{cor}{Corollary}
\newcommand{\BEQA}{\begin{eqnarray}}
\newcommand{\EEQA}{\end{eqnarray}}
\newcommand{\define}{\stackrel{\triangle}{=}}


%\usepackage{scalerel}
%\usepackage{stackengine}
%\newcommand\showdiv[1]{\overline{\smash{\hstretch{.5}{)}\mkern-3.2mu\hstretch{.5}{)}}#1}}
%\let\ph\phantom


\bibliographystyle{IEEEtran}
%\bibliographystyle{ieeetr}

\providecommand{\nCr}[2]{\,^{#1}C_{#2}} % nCr
\providecommand{\nPr}[2]{\,^{#1}P_{#2}} % nPr
\providecommand{\mbf}{\mathbf}
\providecommand{\pr}[1]{\ensuremath{\Pr\left(#1\right)}}
\providecommand{\qfunc}[1]{\ensuremath{Q\left(#1\right)}}
\providecommand{\sbrak}[1]{\ensuremath{{}\left[#1\right]}}
\providecommand{\lsbrak}[1]{\ensuremath{{}\left[#1\right.}}
\providecommand{\rsbrak}[1]{\ensuremath{{}\left.#1\right]}}
\providecommand{\brak}[1]{\ensuremath{\left(#1\right)}}
\providecommand{\lbrak}[1]{\ensuremath{\left(#1\right.}}
\providecommand{\rbrak}[1]{\ensuremath{\left.#1\right)}}
\providecommand{\cbrak}[1]{\ensuremath{\left\{#1\right\}}}
\providecommand{\lcbrak}[1]{\ensuremath{\left\{#1\right.}}
\providecommand{\rcbrak}[1]{\ensuremath{\left.#1\right\}}}
\theoremstyle{remark}
\newtheorem{rem}{Remark}
\newcommand{\sgn}{\mathop{\mathrm{sgn}}}
\providecommand{\abs}[1]{\left\vert#1\right\vert}
\providecommand{\res}[1]{\Res\displaylimits_{#1}} 
\providecommand{\norm}[1]{\lVert#1\rVert}
\providecommand{\mtx}[1]{\mathbf{#1}}
\providecommand{\mean}[1]{E\left[ #1 \right]}
\providecommand{\fourier}{\overset{\mathcal{F}}{ \rightleftharpoons}}
\providecommand{\ztrans}{\overset{\mathcal{Z}}{ \rightleftharpoons}}

%\providecommand{\hilbert}{\overset{\mathcal{H}}{ \rightleftharpoons}}
\providecommand{\system}{\overset{\mathcal{H}}{ \longleftrightarrow}}
	%\newcommand{\solution}[2]{\textbf{Solution:}{#1}}
\newcommand{\solution}{\noindent \textbf{Solution: }}
\providecommand{\dec}[2]{\ensuremath{\overset{#1}{\underset{#2}{\gtrless}}}}
\numberwithin{equation}{section}
%\numberwithin{equation}{subsection}
%\numberwithin{problem}{subsection}
%\numberwithin{definition}{subsection}
\makeatletter
\@addtoreset{figure}{problem}
\makeatother

\let\StandardTheFigure\thefigure
%\renewcommand{\thefigure}{\theproblem.\arabic{figure}}
\renewcommand{\thefigure}{\theproblem}


%\numberwithin{figure}{subsection}

\def\putbox#1#2#3{\makebox[0in][l]{\makebox[#1][l]{}\raisebox{\baselineskip}[0in][0in]{\raisebox{#2}[0in][0in]{#3}}}}
     \def\rightbox#1{\makebox[0in][r]{#1}}
     \def\centbox#1{\makebox[0in]{#1}}
     \def\topbox#1{\raisebox{-\baselineskip}[0in][0in]{#1}}
     \def\midbox#1{\raisebox{-0.5\baselineskip}[0in][0in]{#1}}

\vspace{3cm}

\title{
%\logo{
Digital Signal Processing
%}
%	\logo{Octave for Math Computing }
}
%\title{
%	\logo{Matrix Analysis through Octave}{\begin{center}\includegraphics[scale=.24]{tlc}\end{center}}{}{HAMDSP}
%}


% paper title
% can use linebreaks \\ within to get better formatting as desired
%\title{Matrix Analysis through Octave}
%
%
% author names and IEEE memberships
% note positions of commas and nonbreaking spaces ( ~ ) LaTeX will not break
% a structure at a ~ so this keeps an author's name from being broken across
% two lines.
% use \thanks{} to gain access to the first footnote area
% a separate \thanks must be used for each paragraph as LaTeX2e's \thanks
% was not built to handle multiple paragraphs
%

\author{ Himanshu Kumar Gupta%<-this  stops a space
	
% <-this % stops a space
%\thanks{J. Doe and J. Doe are with Anonymous University.}% <-this % stops a space
%\thanks{Manuscript received April 19, 2005; revised January 11, 2007.}}
}
% note the % following the last \IEEEmembership and also \thanks - 
% these prevent an unwanted space from occurring between the last author name
% and the end of the author line. i.e., if you had this:
% 
% \author{....lastname \thanks{...} \thanks{...} }
%                     ^------------^------------^----Do not want these spaces!
%
% a space would be appended to the last name and could cause every name on that
% line to be shifted left slightly. This is one of those "LaTeX things". For
% instance, "\textbf{A} \textbf{B}" will typeset as "A B" not "AB". To get
% "AB" then you have to do: "\textbf{A}\textbf{B}"
% \thanks is no different in this regard, so shield the last } of each \thanks
% that ends a line with a % and do not let a space in before the next \thanks.
% Spaces after \IEEEmembership other than the last one are OK (and needed) as
% you are supposed to have spaces between the names. For what it is worth,
% this is a minor point as most people would not even notice if the said evil
% space somehow managed to creep in.



% The paper headers
%\markboth{Journal of \LaTeX\ Class Files,~Vol.~6, No.~1, January~2007}%
%{Shell \MakeLowercase{\textit{et al.}}: Bare Demo of IEEEtran.cls for Journals}
% The only time the second header will appear is for the odd numbered pages
% after the title page when using the twoside option.
% 
% *** Note that you probably will NOT want to include the author's ***
% *** name in the headers of peer review papers.                   ***
% You can use \ifCLASSOPTIONpeerreview for conditional compilation here if
% you desire.




% If you want to put a publisher's ID mark on the page you can do it like
% this:
%\IEEEpubid{0000--0000/00\$00.00~\copyright~2007 IEEE}
% Remember, if you use this you must call \IEEEpubidadjcol in the second
% column for its text to clear the IEEEpubid mark.

\newcommand{\myvec}[1]{\ensuremath{\begin{pmatrix}#1\end{pmatrix}}}

\let\vec\mathbf

% make the title area
\maketitle

%\newpage

\tableofcontents

%\renewcommand{\thefigure}{\thesection.\theenumi}
%\renewcommand{\thetable}{\thesection.\theenumi}

\renewcommand{\thefigure}{\theenumi}
\renewcommand{\thetable}{\theenumi}

%\renewcommand{\theequation}{\thesection}


\bigskip

\begin{abstract}
This manual provides a simple introduction to digital signal processing.
\end{abstract}
\section{Software Installation}
Run the following commands
\begin{lstlisting}
sudo apt-get update
sudo apt-get install libffi-dev libsndfile1 python3-scipy  python3-numpy python3-matplotlib 
sudo pip install cffi pysoundfile 
\end{lstlisting}
\section{Digital Filter}
\begin{enumerate}[label=\thesection.\arabic*
,ref=\thesection.\theenumi]
\item
\label{prob:input}
Download the sound file from  
\begin{lstlisting}
wget https://github.com/himanshukumargupta11012/EE3900_assignments/blob/master/assignment_1/ques_2/Sound_Noise.wav
\end{lstlisting}
%\href{http://tlc.iith.ac.in/img/sound/Sound_Noise.wav}{\url{http://tlc.iith.ac.in/img/sound/Sound_Noise.wav}}  
%in the link given below.
%\linebreak
\item
\label{prob:spectrogram}
You will find a spectrogram at \href{https://academo.org/demos/spectrum-analyzer}{\url{https://academo.org/demos/spectrum-analyzer}}. 
%\end{problem}
%%
%
%%\onecolumn
%%\input{./figs/fir}
%\begin{problem}
Upload the sound file that you downloaded in Problem \ref{prob:input} in the spectrogram  and play.  Observe the spectrogram. What do you find?
\\
%
\solution There are a lot of yellow lines between 440 Hz to 5.1 KHz.  These represent the synthesizer key tones. Also, the key strokes
are audible along with background noise.
% By observing spectrogram, it clearly shows that tonal frequency is under 4kHz. And above 4kHz only noise is present.
\item
\label{prob:output}
Write the python code for removal of out of band noise and execute the code.
\\
\solution
\lstinputlisting{./ques_2/2.3.py}
%\begin{figure}[h]
%\centering
%\includegraphics[width=\columnwidth]{enc_block_diag.png}
%\caption{}
%\label{fig:convolution encoder}
%\end{figure}
%\input{block_enc}
\item
The output of the python script in Problem \ref{prob:output} is the audio file Sound\_Without\_Noise.wav. Play the file in the spectrogram in Problem \ref{prob:spectrogram}. What do you observe?
\\
\solution The key strokes as well as background noise is subdued in the audio.  Also,  the signal is blank for frequencies above 5.1 kHz.

\end{enumerate}
\section{Difference Equation}
\begin{enumerate}[label=\thesection.\arabic*,ref=\thesection.\theenumi]
\item Let
\label{def:x_n}
\begin{equation}
x(n) = \cbrak{\underset{\uparrow}{1},2,3,4,2,1}
\end{equation}
Sketch $x(n)$.

\solution The following code yields Fig. \ref{fig:x_n}.
\begin{lstlisting}
wget https://github.com/himanshukumargupta11012/EE3900_assignments/blob/master/assignment_1/ques_3/3.1_2.py
\end{lstlisting}
\begin{figure}[!ht]
	\begin{center}
		\includegraphics[width=\columnwidth]{./ques_3/x_n.png}
	\end{center}
	\captionof{figure}{$x(n)$ wrt n}
	\label{fig:x_n}	
\end{figure}
\item Let
\begin{multline}
\label{eq:iir_filter}
y(n) + \frac{1}{2}y(n-1) = x(n) + x(n-2), 
\\
 y(n) = 0, n < 0
\end{multline}
Sketch $y(n)$.
\\
\solution The following code yields Fig. \ref{fig:y_n}.
\begin{lstlisting}
wget https://github.com/himanshukumargupta11012/EE3900_assignments/blob/master/assignment_1/ques_3/3.1_2.py
\end{lstlisting}
\begin{figure}[!ht]
\begin{center}
\includegraphics[width=\columnwidth]{./ques_3/y_n.png}
\end{center}
\captionof{figure}{y(n) wrt n}
\label{fig:y_n}	
\end{figure}
\item Repeat the above exercise using a C code.

\solution Download and run the following code 
\begin{lstlisting}
wget https://github.com/himanshukumargupta11012/EE3900_assignments/blob/master/assignment_1/ques_3/3.3.c
\end{lstlisting}
\end{enumerate}
\section{$Z$-transform}
\begin{enumerate}[label=\thesection.\arabic*]
\item The $Z$-transform of $x(n)$ is defined as
%
\begin{equation}
\label{eq:z_trans}
X(z)={\mathcal {Z}}\{x(n)\}=\sum _{n=-\infty }^{\infty }x(n)z^{-n}
\end{equation}
%
Show that
\begin{equation}
\label{eq:shift1}
{\mathcal {Z}}\{x(n-1)\} = z^{-1}X(z)
\end{equation}
and find
\begin{equation}
	{\mathcal {Z}}\{x(n-k)\} 
\end{equation}
\solution From \eqref{eq:z_trans}
\begin{align}
	{\mathcal {Z}}\{x(n-1)\}=\sum_{n=-\infty}^\infty x(n-1)z^{-n}
\end{align}
Let $m=n-1$.Then
\begin{align}
	{\mathcal {Z}}\{x(n-1)\}&=\sum_{m+1=-\infty}^\infty x(m)z^{-m-1}\\
	&=z^{-1}\sum_{m=-\infty}^\infty x(m)z^{-m}\\
	&=z^{-1}X(z)
\end{align}
%\solution From \eqref{eq:z_trans},
%\begin{align}
%{\mathcal {Z}}\{x(n-k)\} &=\sum _{n=-\infty }^{\infty }x(n-1)z^{-n}
%\\
%&=\sum _{n=-\infty }^{\infty }x(n)z^{-n-1} = z^{-1}\sum _{n=-\infty }^{\infty }x(n)z^{-n}
%\end{align}
Similarly, it can be shown that
%
\begin{equation}
\label{eq:z_trans_shift}
	{\mathcal {Z}}\{x(n-k)\} = z^{-k}X(z)
\end{equation}


\item Obtain $X(z)$ for $x(n)$ defined in problem 
\ref{def:x_n}.

\solution $Z$-transform of x(n),$X(z)$ is given by
\begin{align}
	X(n)&=\sum_{n=-\infty}^\infty x(n)z^{-n}\\
	&=\sum_{n=0}^5x(n)z^{-n}\\
	&=1+2z^{-1}+3z^{-2}+4z^{-3}+2z^{-4}+z^{-5}
\end{align}
\item Find
%
\begin{equation}
H(z) = \frac{Y(z)}{X(z)}
\end{equation}
%
from  \eqref{eq:iir_filter} assuming that the $Z$-transform is a linear operation.

%\solution Applying $Z$-transform to \eqref{eq:iir_filter}
%\begin{align}
%	y(n) + \frac{1}{2}y(n-1) = x(n) + x(n-2)
%\end{align}

	
\solution  Applying \eqref{eq:z_trans_shift} in \eqref{eq:iir_filter},
\begin{align}
Y(z) + \frac{1}{2}z^{-1}Y(z) &= X(z)+z^{-2}X(z)
\\
\implies \frac{Y(z)}{X(z)} &= \frac{1 + z^{-2}}{1 + \frac{1}{2}z^{-1}}
\label{eq:freq_resp}
\end{align}
%gggggggggggggggggggggggggggggggggggggggggggggggggggggggggggggggg
\item Find the Z transform of 
\begin{equation}
\delta(n)
=
\begin{cases}
1 & n = 0
\\
0 & \text{otherwise}
\end{cases}
\end{equation}
and show that the $Z$-transform of
\begin{equation}
\label{eq:unit_step}
u(n)
=
\begin{cases}
1 & n \ge 0
\\
0 & \text{otherwise}
\end{cases}
\end{equation}
is
\begin{equation}
U(z) = \frac{1}{1-z^{-1}}, \quad \abs{z} > 1
\end{equation}
%\solution It is easy to show that
%\begin{equation}
%\delta(n) \ztrans 1
%\end{equation}
\solution $Z$-transform of $\delta(n)$ is given by
\begin{align}
	\mathcal{Z}\{\delta(n)\}&=\sum _{n= -\infty}^{\infty}\delta(n)z^{-n}\\
	&=1
\end{align} 
and from \eqref{eq:unit_step},
\begin{align}
U(z) &= \sum _{n= 0}^{\infty}z^{-n}
\\
&=\frac{1}{1-z^{-1}}, \quad \abs{z} > 1
\end{align}
using the fomula for the sum of an infinite geometric progression.
%
\item Show that 
\begin{equation}
\label{eq:anun}
a^nu(n) \ztrans \frac{1}{1-az^{-1}} \quad \abs{z} > \abs{a}
\end{equation}

\solution $Z$-transform is given by
\begin{align}
	\mathcal{Z}\{a^nu(n)\}&=\sum_{n=-\infty}^\infty a^nu(n)z^{-n}\\
	&=\sum_{n=0}^\infty (az^{-1})^{n}\\
	&=\frac{1}{1-az^{-1}},\quad \abs{z} > \abs{a}
\end{align}
\item 
Let
\begin{equation}
H\brak{e^{\j \omega}} = H\brak{z = e^{\j \omega}}.
\end{equation}
Plot $\abs{H\brak{e^{\j \omega}}}$.  Comment.Is it periodic? If so, find the
period.  $H(e^{\j \omega})$ is
known as the {\em Discret Time Fourier Transform} (DTFT) of $h(n)$.
\\
\solution $H(e^{jw})$ is given by
\begin{align}
	H(e^{j\omega})&=\frac{1+(e^{j\omega})^{-2}}{1+\frac{1}{2}(e^{j\omega})^{-1}}\\
	&=2\frac{1+\cos(-2\omega)+j\sin(-2\omega)}{2+\cos(-\omega)+j\sin(-\omega)}\\
	&=2\frac{1+\cos(2\omega)-j\sin(2\omega)}{2+\cos(\omega)-j\sin(\omega)}\\
	&=2\frac{2\cos^2(\omega)-2j\sin(\omega)\cos(\omega)}{2+\cos(\omega)-j\sin(\omega)}\\
	&=4\cos(\omega)\frac{\cos(\omega)-j\sin(\omega)}{2+\cos(\omega)-j\sin(\omega)}
\end{align}
So,
\begin{align}
	|H(e^{j\omega})|=\frac{4|\cos(\omega)|}{\sqrt{5+4\cos(\omega)}}
\end{align}
\begin{align}
|H(e^{j(\omega+2\pi})|&=\frac{4|\cos(\omega+2\pi)|}{\sqrt{5+4\cos(\omega+2\pi)}}\\
&=\frac{4|\cos(\omega)|}{\sqrt{5+4\cos(\omega)}}\\
&=|H(e^{j\omega})|
\end{align}
Yes it is periodic because $\cos$ is periodic function.
Period of numerator is $\pi$ and period of denominator is $2\pi$
So,period of $|H(e^{jw})|$ would be LCM of $\pi$ and $2\pi$ which is $2\pi$.

Same you can see graphcally also.
 
The following code plots Fig. \ref{fig:dtft}.
\begin{lstlisting}
wget https://github.com/himanshukumargupta11012/EE3900_assignments/blob/master/assignment_1/ques_4/4.5.py
\end{lstlisting}
\begin{figure}[!ht]
\centering
\includegraphics[width=\columnwidth]{./ques_4/H_w}
\caption{Discret Time Fourier Transform}
\label{fig:dtft}
\end{figure}
\item Express $h(n)$ in terms of $H\brak{e^{\j \omega}}$.

\solution We know that 
\begin{align}
	H(e^{j\omega})=\sum_{k=-\infty}^\infty h(k)e^{-j\omega k}
\end{align}
and
\begin{align}
	h(n)=\frac{1}{2\pi}\int_{-\pi}^\pi H(e^{j\omega})e^{j\omega n}d\omega 
\end{align}
Now,
\begin{align}
\frac{1}{2\pi}\int_{-\pi}^\pi& H(e^{j\omega})e^{j\omega n}d\omega\\
 &=\frac{1}{2\pi}\int_{-\pi}^\pi \sum_{k=-\infty}^\infty h(k)e^{-j\omega k}e^{j\omega n}d\omega\\
 &=\frac{1}{2\pi} \sum_{k=-\infty}^\infty h(k)\int_{-\pi}^\pi e^{j\omega (n-k)}d\omega\\
% &=\frac{1}{2\pi}\cbrak{ \sum_{k\neq n} h(k)\frac{e^{j\omega (n-k)}}{j(n-k)}\Biggr] _{-\pi}^\pi 	+h(n)\int_{-\pi}^\pi d\omega }\\
 &=\frac{1}{2\pi}\sum_{k\neq n} h(k)\frac{e^{j\omega (n-k)}}{j(n-k)}\Biggr] _{-\pi}^\pi \\
 &	+\frac{1}{2\pi}h(n)\int_{-\pi}^\pi d\omega \\
 &=\frac{0+2\pi h(n)}{2\pi}\\
 &=h(n)
\end{align}

\end{enumerate}

\section{Impulse Response}
\begin{enumerate}[label=\thesection.\arabic*]
	\item Using long division, 
	find
	\begin{align}
		h(n), \quad n < 5
	\end{align}
	for H(z) in 
	\eqref{eq:freq_resp}.


\solution $H(z)$ is given by
\begin{align}
	H(z)=\frac{1+z^{-2}}{1+\frac{1}{2}z^{-1}}=\frac{2+2z^{-2}}{2+z^{-1}}
\end{align}
Now, on doing long division,
\begin{align}
	&2z^{-1}-4    \nonumber\\	
  z^{-1}+2\hspace{2mm}&\overline{\big)\hspace{2mm} 2z^{-2}+2\hspace{15mm}}   \nonumber\\
	& \hspace{2mm} 2z^{-2}+4z^{-1}   \nonumber\\
	&\overline{\hspace{11mm}-4z^{-1}+2\hspace{5mm}}   \nonumber\\
	&\hspace{9mm}-4z^{-1}-8   \nonumber\\ 
	&\overline{\hspace{24mm}10}\nonumber
\end{align}
So,
\begin{align}
	H(z)&=2z^{-1}-4+\frac{10}{z^{-1}+2}\\
	&=2z^{-1}-4+\frac{5}{\frac{1}{2}z^{-1}+1}\\
	&=2z^{-1}-4+5\sum_{n=0}^\infty \brak{-\frac{z^{-1}}{2}}^{n}\\
	&=1-\frac{1}{2}z^{-1}+\sum_{n=2}^\infty \brak{-\frac{1}{2}}^{n}z^{-n}
\end{align}
So,h(n) will be given by 
\begin{align}
	h(n)=\begin{cases}
		\label{eq:h_n_def}
		5\times \brak{-\frac{1}{2}}^n  & n\geq 2\\
		\brak{-\frac{1}{2}}^n  &2>n\geq0\\
		0 &n<0
	\end{cases}
\end{align}
%So,now
%\begin{align}
%	&h(0)=1\\
%	&h(1)=-\frac{1}{2}\\
%	&h(2)=\frac{5}{4}\\
%	&h(3)=-\frac{5}{8}\\
%	&h(4)=\frac{5}{16}
%\end{align}
\item \label{prob:impulse_resp}
Find an expression for $h(n)$ using $H(z)$, given that 
%in Problem \ref{eq:ztransab} and \eqref{eq:anun}, given that
\begin{equation}
\label{eq:impulse_resp}
h(n) \ztrans H(z)
\end{equation}
and there is a one to one relationship between $h(n)$ and $H(z)$. $h(n)$ is known as the {\em impulse response} of the
system defined by \eqref{eq:iir_filter}.
\\
\solution From \eqref{eq:freq_resp},
\begin{align}
H(z) &= \frac{1}{1 + \frac{1}{2}z^{-1}} + \frac{ z^{-2}}{1 + \frac{1}{2}z^{-1}}
\\
\implies h(n) &= \brak{-\frac{1}{2}}^{n}u(n) + \brak{-\frac{1}{2}}^{n-2}u(n-2)
\end{align}
using \eqref{eq:anun} and \eqref{eq:z_trans_shift}.
\item Sketch $h(n)$. Is it bounded? Justify theoreti-
cally.

\solution The following code plots Fig. \ref{fig:h_n_u}.

\begin{lstlisting}
wget https://github.com/himanshukumargupta11012/EE3900_assignments/blob/master/assignment_1/ques_5/5.2.py
\end{lstlisting}
Graphically we can see that $h(n)$ is bounded at starting and its decreasing as n increases.So,we can conclude that it is bounded.

$h(n)$ is given by
\begin{align}
	h(n)=
	\begin{cases}
		\label{eq:h_n_def}
		5\times \brak{-\frac{1}{2}}^n  & n\geq 2\\
		\brak{-\frac{1}{2}}^n  &2>n\geq0\\
		0 &n<0
	\end{cases}
\end{align}
A sequence \{$x_n$\} is said to be bounded if and only if there exist a positive real number N such that 
\begin{align}
	|x_i|\leq N \quad\forall i\in \mathbb{N}
\end{align} 
for $n<0$:
\begin{align}
	|h(n)|\leq 0
\end{align}
for $0\leq n<2$:
\begin{align}
	|h(n)|=\abs{-\frac{1}{2}}^n=\brak{\frac{1}{2}}^n \leq 1
\end{align}
for $2 \leq n$
\begin{align}
	|h(n)|=5\abs{-\frac{1}{2}}^n=5\brak{\frac{1}{2}}^n \leq \frac{5}{4}
\end{align}
Combining all we get 
\begin{align}
	\abs{h(n)}\leq \frac{5}{4}
\end{align}
So,we can conclude that h(n) is bounded

\begin{figure}[!ht]
\centering
\includegraphics[width=\columnwidth]{./ques_5/h_n}
\caption{$h(n)$ wrt n}
\label{fig:h_n_u}
\end{figure}
%
\item Convergent? Justify using the ratio test.

\solution According to ratio test, a sequence  \{$x_n$\} is convergent if 
\begin{align}
	\lim_{n \to \infty }\abs{\frac{x^{n+1}}{x^n}}<1
\end{align}

\begin{align}
\because \lim_{n \to \infty} \abs{\frac{h(n+1)}{h(n)}}&=\abs{\frac{5\times \brak{-\frac{1}{2}}^{(n+1)}}{5\times \brak{-\frac{1}{2}}^n} }\\
&=\frac{1}{2}
\end{align}
Therefore,$h(n)$ is convergent
\item The system with $h(n)$ is defined to be stable if
\begin{equation}
\sum_{n=-\infty}^{\infty}h(n) < \infty
\end{equation}
Is the system defined by \eqref{eq:iir_filter} stable for the impulse response in \eqref{eq:impulse_resp}?

\solution For system of 3.2 ,$h(n)$ is defined in \eqref{eq:h_n_def} 
So,
\begin{align}
	\sum_{n=-\infty}^{\infty}h(n)&=\sum_{n=2}^{\infty}5\times \brak{-\frac{1}{2}}^n+\sum_{n=0}^{1} \brak{-\frac{1}{2}}^n+\sum_{n=-\infty}^{-1}0\\
	&=5\times \frac{1}{6}+\frac{1}{2}\\
	&=\frac{4}{3}
\end{align}
Since the sum is finite so the system is stable for impulsive response
\item Verify the above result using a python code.

\solution Download and run the below python code .
\begin{lstlisting}
wget https://github.com/himanshukumargupta11012/EE3900_assignments/blob/master/assignment_1/ques_5/stable.py
\end{lstlisting}
\item 
Compute and sketch $h(n)$ using 
\begin{equation}
\label{eq:iir_filter_h}
h(n) + \frac{1}{2}h(n-1) = \delta(n) + \delta(n-2), 
\end{equation}
%
This is the definition of $h(n)$.
\\
\solution The following code plots Fig. \ref{fig:h_n_delta}. Note that this is the same as Fig. 
\ref{fig:h_n_u}. 
%
\begin{lstlisting}
wget https://github.com/himanshukumargupta11012/EE3900_assignments/blob/master/assignment_1/ques_5/5.4.py
\end{lstlisting}
\begin{figure}[!ht]
\centering
\includegraphics[width=\columnwidth]{./ques_5/h_delta}
\caption{$h(n)$ from the definition}
\label{fig:h_n_delta}
\end{figure}
%
\item Compute 
%
\begin{equation}
\label{eq:convolution}
y(n) = x(n)*h(n) = \sum_{k=-\infty}^{\infty}x(k)h(n-k)
\end{equation}
%
Comment. The operation in \eqref{eq:convolution} is known as
{\em convolution}.
%
\\
\solution The following code plots Fig. \ref{fig:y_n_convo}. Note that this is the same as 
$y(n)$ in  Fig. 
\ref{fig:y_n}. 


\begin{lstlisting}
wget https://github.com/himanshukumargupta11012/EE3900_assignments/blob/master/assignment_1/ques_5/5.5.py
\end{lstlisting}
\begin{figure}[!ht]
\centering
\includegraphics[width=\columnwidth]{./ques_5/y_n_convo}
\caption{$y(n)$ from the definition of convolution}
\label{fig:y_n_convo}
\end{figure}

\item Express the above convolution using a Teoplitz matrix.

\solution
For finding the above convolution using topleitz matrix we have to find topleitz matrix of h(n).

h(n) is tending to 0 for large n.
So,we take upto some n only.
So,
\begin{align}
	h(n)=\myvec{1\\-.5\\.\\.} \quad for \quad n=0,2 ...9
\end{align} 
So,topleitz matrix of $h(n)$ will be
\begin{align}
	top\{h(n)\}=\myvec{1& 0& 0 &0 &0 &0\\-.5 &1 &0 &0 &0 &0\\1.25&-.5&1&0&0&0\\. &.& .& .& .&. }
\end{align}
and 
\begin{align}
	x(n)=\myvec{1\\2\\3\\4\\2\\1}
\end{align}
So,
\begin{align}
	x(n)*h(n)&=top\{h(n)\}x(n)\\
	&=\myvec{1\\1.5\\3.25\\.\\.}
\end{align}
\item Show that
\begin{equation}
y(n) =  \sum_{k=-\infty}^{\infty}x(n-k)h(k)
\end{equation}

\solution From \eqref{eq:convolution}
\begin{align}
	y(n) = \sum_{k=-\infty}^{\infty}x(k)h(n-k)
\end{align}
Replacing n-k with l,we get
\begin{align}
	y(n) &= \sum_{n-l=-\infty}^{\infty}x(n-l)h(l)\\
	&=\sum_{-l=-\infty}^{\infty}x(n-l)h(l)\\
		&=\sum_{l=-\infty}^{\infty}x(n-l)h(l)
\end{align}
\end{enumerate}

%
\section{DFT}
\begin{enumerate}[label=\thesection.\arabic*]
\item
Compute
\begin{equation}
X(k) \define \sum _{n=0}^{N-1}x(n) e^{-\j2\pi kn/N}, \quad k = 0,1,\dots, N-1
\end{equation}
and $H(k)$ using $h(n)$.

\solution Download and run the following code.
\begin{lstlisting}
wget https://github.com/himanshukumargupta11012/EE3900_assignments/blob/master/assignment_1/ques_6/dtft_sum.py
\end{lstlisting}
\begin{figure}[!ht]
	\centering
	\includegraphics[width=\columnwidth]{./ques_6/dtft_sum}
	\title{X(k) and H(k) graph wrt k}
	\label{fig:dtft_sum}
\end{figure}
\item Compute 
\begin{equation}
Y(k) = X(k)H(k)
\end{equation}

\begin{figure}[!ht]
	\centering
	\includegraphics[width=\columnwidth]{./ques_6/Y_k}
	\caption{$Y(k)$ using $X(k)$ and $H(k)$}
	\label{fig:Y_k}
\end{figure}
\item Compute
\begin{equation}
 y\brak{n}={\frac {1}{N}}\sum _{k=0}^{N-1}Y\brak{k}\cdot e^{\j 2\pi kn/N},\quad n = 0,1,\dots, N-1
\end{equation}

\solution The following code plots Fig. \ref{fig:y_n_idtft}. Note that this is the same as 
$y(n)$ in  Fig. 
\ref{fig:y_n}. 
%
\begin{lstlisting}
wget https://github.com/himanshukumargupta11012/EE3900_assignments/blob/master/assignment_1/ques_6/y_n_idtft.py
\end{lstlisting}
\begin{figure}[!ht]
\centering
\includegraphics[width=\columnwidth]{./ques_6/y_n_idtft}
\caption{$y(n)$ from the DFT}
\label{fig:y_n_idtft}
\end{figure}

\item Repeat the previous exercise by computing $X(k), H(k)$ and $y(n)$ through FFT and 
IFFT.
\solution Download and rum the following code.
\begin{lstlisting}
wget https://github.com/himanshukumargupta11012/EE3900_assignments/blob/master/assignment_1/ques_6/fft_ifft.py
\end{lstlisting}
\begin{figure}[!ht]
	\centering
	\includegraphics[width=\columnwidth]{./ques_6/X_H_fft.png}
	\caption{$H(k)$ and $X(k)$ using FFT}
	\label{fig:X_h_fft}
\end{figure}
\begin{figure}[!ht]
	\centering
	\includegraphics[width=\columnwidth]{./ques_6/Y_fft}
	\caption{$Y(k)$ using $X(k)$ and $H(k)$}
	\label{fig:Y_fft}
\end{figure}
\begin{figure}[!ht]
	\centering
	\includegraphics[width=\columnwidth]{./ques_6/y_ifft}
	\caption{comparison between $y(n)$ coming by different methods}
	\label{fig:y_ifft}
\end{figure}
\end{enumerate}

\section{FFT}
% \subsection{Definitions}
\begin{enumerate}[label=\arabic*.,ref=\thesection.\theenumi]
	\numberwithin{equation}{section}
	\item The DFT of $x(n)$ is given by
	\begin{align}
		X(k) \triangleq \sum_{n=0}^{N-1} x(n) e^{-j 2 \pi k n / N}, \quad k=0,1, \ldots, N-1
	\end{align}
	\item Let 
	\begin{align}
		W_{N} = e^{-j2\pi/N} 
	\end{align}
	Then the $N$-point {\em DFT matrix} is defined as 
	\begin{align}
		\vec{F}_{N} = \sbrak{W_{N}^{mn}}, \quad 0 \le m,n \le N-1 
	\end{align}
	where $W_{N}^{mn}$ are the elements of $\vec{F}_{N}$.
	\item Let 
	\begin{align}
		\vec{I}_4 = \myvec{\vec{e}_4^{1} &\vec{e}_4^{2} &\vec{e}_4^{3} &\vec{e}_4^{4} }
	\end{align}
	be the $4\times 4$ identity matrix.  Then the 4 point {\em DFT permutation matrix} is defined as 
	\begin{align}
		\vec{P}_4 = \myvec{\vec{e}_4^{1} &\vec{e}_4^{3} &\vec{e}_4^{2} &\vec{e}_4^{4} }
	\end{align}
	\item The 4 point {\em DFT diagonal matrix} is defined as 
	\begin{align}
		\vec{D}_4 = diag\myvec{W_{8}^{0} & W_{8}^{1} & W_{8}^{2} & W_{8}^{3}}
	\end{align}
	\item Show that 
	\begin{equation}
		W_{N}^{2}=W_{N/2}
	\end{equation}


\solution 
\begin{align}
	W_N^2&=\left(e^{-j2\pi/N}\right)^2\\
	&=e^{-j4\pi/N}\\
	&=e^{-\frac{j2\pi}{N/2}}\\
	&=W_{N/2}
\end{align}
	%    \item Find $\vec{P}_6$.
	%    \item Find $\vec{D}_3$.
	\item Show that 
	\begin{equation}
		\vec{F}_{4}=
		\begin{bmatrix}
			\vec{I}_{2} & \vec{D}_{2} \\
			\vec{I}_{2} & -\vec{D}_{2}
		\end{bmatrix}
		\begin{bmatrix}
			\vec{F}_{2} & 0 \\
			0 & \vec{F}_{2}
		\end{bmatrix}
		\vec{P}_{4}
	\end{equation}
\solution 
\begin{align}
	\vec{F}_{4}&=
	\begin{bmatrix}
		\vec{I}_{2} & \vec{D}_{2} \\
		\vec{I}_{2} & -\vec{D}_{2}
	\end{bmatrix}
	\begin{bmatrix}
		\vec{F}_{2} & 0 \\
		0 & \vec{F}_{2}
	\end{bmatrix}
	\vec{P}_{4}\\
	\implies \vec{F}_4\vec{P}_4&=\begin{bmatrix}
		\vec{I}_{2} & \vec{D}_{2} \\
		\vec{I}_{2} & -\vec{D}_{2}
	\end{bmatrix}
	\begin{bmatrix}
		\vec{F}_{2} & 0 \\
		0 & \vec{F}_{2}
	\end{bmatrix}
	\vec{P}_{4}^2\\
	\implies \vec{F}_4\vec{P}_4&=\begin{bmatrix}
		\vec{F}_{2} & \vec{D}_{2}\vec{F}_{2} \\
		\vec{F}_{2} & -\vec{D}_{2}\vec{F}_{2}
	\end{bmatrix}
\end{align}
Now,on multiplying $\vec{F}_4$ and $\vec{P}_4$ , we get
\begin{align}
	&\vec{F}_4\vec{P}_4 =\begin{bmatrix}
		W_4^{0\times0}&W_4^{0\times2}&W_4^{0\times1}&W_4^{0\times3}\\
		W_4^{1\times0}&W_4^{1\times2}&W_4^{1\times1}&W_4^{1\times3}\\
		W_4^{2\times0}&W_4^{2\times2}&W_4^{2\times1}&W_4^{2\times3}\\
		W_4^{3\times0}&W_4^{3\times2}&W_4^{3\times1}&W_4^{3\times3}
	\end{bmatrix}\\
&=\begin{bmatrix}
	\left(W_4^{0\times0}\right)^2&\left(W_4^{0\times1}\right)^2 \\
	\left(W_4^{1\times0}\right)^2&\left(W_4^{1\times1}\right)^2&\vec{M}\\
	\left(W_4^{2\times0}\right)^2&\left(W_4^{2\times1}\right)^2\\
	\left(W_4^{3\times0}\right)^2&\left(W_4^{3\times1}\right)^2
\end{bmatrix}\\
&=\begin{bmatrix}
	\left(W_4^{0\times0}\right)^2&\left(W_4^{0\times1}\right)^2\\
	\left(W_4^{1\times0}\right)^2&\left(W_4^{1\times1}\right)^2&\vec{M}\\
	\left((-1)^0W_4^{\left(2-\frac{4}{2}\right)\times0}\right)^2&\left((-1)^1W_4^{\left(2-\frac{4}{2}\right)\times1}\right)^2\\
	\left((-1)^0W_4^{\left(3-\frac{4}{2}\right)\times0}\right)^2&\left((-1)^1W_4^{\left(3-\frac{4}{2}\right)\times1}\right)^2
\end{bmatrix}\\
&=\begin{bmatrix}
	\left(W_4^{0\times0}\right)^2&\left(W_4^{0\times1}\right)^2&W_4^0\left(W_4^{0\times0}\right)&W_4^0\left(W_4^{0\times2}\right)\\
	\left(W_4^{1\times0}\right)^2&\left(W_4^{1\times1}\right)^2&W_4^1\left(W_4^{1\times0}\right)&W_4^1\left(W_4^{1\times2}\right)\\
	\left(W_4^{0\times0}\right)^2&\left(W_4^{0\times1}\right)^2&W_4^2\left(W_4^{2\times0}\right)&W_4^2\left(W_4^{2\times2}\right)\\
	\left(W_4^{1\times0}\right)^2&\left(W_4^{1\times1}\right)^2&W_4^3\left(W_4^{3\times0}\right)&W_4^3\left(W_4^{3\times2}\right)
\end{bmatrix}\\
&=\begin{bmatrix}
	W_2^{0\times0}&W_2^{0\times1}&W_4^0\left(W_4^{0\times0}\right)&W_4^0\left(W_4^{0\times2}\right)\\
	W_2^{1\times0}&W_2^{1\times1}&W_4^1\left(W_4^{1\times0}\right)&W_4^1\left(W_4^{1\times2}\right)\\
	W_2^{0\times0}&W_2^{0\times1}&-W_4^{2-\frac{4}{2}}\left(W_4^{2\times0}\right)&-W_4^{2-\frac{4}{2}}\left(W_4^{2\times2}\right)\\
	W_2^{1\times0}&W_2^{1\times1}&-W_4^{3-\frac{4}{2}}\left(W_4^{3\times0}\right)&-W_4^{3-\frac{4}{2}}\left(W_4^{3\times2}\right)
\end{bmatrix}\\
&=\begin{bmatrix}
	\vec{F}_2&W_4^0W_2^{0\times0}&W_4^0W_2^{0\times1}\\
	&W_4^1W_2^{1\times0}&W_4^1W_2^{1\times1}\\
	\vec{F}_2&-W_4^0W_2^{0\times0}&-W_4^0W_2^{0\times1}\\
	&-W_4^1W_2^{1\times0}&-W_4^1W_2^{1\times1}
\end{bmatrix}\\
&=\begin{bmatrix}
	\vec{F}_2&\vec{D}_2\vec{F}_2\\
	\vec{F}_2&-\vec{D}_2\vec{F}_2
\end{bmatrix}
\end{align}
	\item Show that 
	\begin{equation}
		\vec{F}_{N}=
		\begin{bmatrix}
			\vec{I}_{N/2} & \vec{D}_{N/2} \\
			\vec{I}_{N/2} & -\vec{D}_{N/2}
		\end{bmatrix}
		\begin{bmatrix}
			\vec{F}_{N/2} & 0 \\
			0 & \vec{F}_{N/2}
		\end{bmatrix}
		\vec{P}_{N}
	\end{equation}
\solution For N  as an even number,
\begin{align}
	&\vec{F}_N\vec{P}_N=\begin{bmatrix}
		W_N^{0\times0}&W_N^{0\times2}&..&W_N^{0\times1}&W_N^{0\times3}&..\\
		W_N^{1\times0}&W_N^{1\times2}&..&W_N^{1\times1}&W_N^{1\times3}&..\\
		..&..&..&..&..&..\\
		W_N^{N/2\times0}&W_N^{N/2\times2}&..&W_N^{N/2\times1}&W_N^{N/2\times3}&..\\
		..&..&..&..&..&..\\
		W_N^{N-1\times0}&W_N^{0\times2}&..&W_N^{N-1\times1}&W_N^{N-1\times3}&..\\
	\end{bmatrix}\\
&=\begin{bmatrix}
	\begin{bmatrix}
		W_N^{n\times 2m}
	\end{bmatrix}&\begin{bmatrix}
	W_N^{n\times (2m+1)}
\end{bmatrix}\\
%\begin{bmatrix}
%	
%\end{bmatrix}
\begin{bmatrix}
	W_N^{(n+N/2)\times 2m}
\end{bmatrix}&
\begin{bmatrix}
	W_N^{(n+N/2)\times (2m+1)}
\end{bmatrix}	
\end{bmatrix}\\
&\hspace{40mm}\text{where } 0 \le m,n \le \frac{N}{2}-1  \nonumber\\
&=\begin{bmatrix}
	\begin{bmatrix}
		W_N^{n\times 2m}
	\end{bmatrix}&\begin{bmatrix}
	W_N^{n\times (2m+1)}
\end{bmatrix}\\
	\begin{bmatrix}
		W_N^{n\times 2m+\frac{N}{2}\times 2m}
	\end{bmatrix}&\begin{bmatrix}
	W_N^{n\times (2m+1)+\frac{N}{2}\times (2m+1)}
\end{bmatrix}
\end{bmatrix}\\
&=\begin{bmatrix}
	\begin{bmatrix}
		W_N^{n\times 2m}
	\end{bmatrix}&
\begin{bmatrix}
	W_N^{n\times (2m+1)}
\end{bmatrix}\\
	\begin{bmatrix}
		W_N^{n\times 2m}
	\end{bmatrix}&
\begin{bmatrix}
	-W_N^{n\times (2m+1)}
\end{bmatrix}
\end{bmatrix}\\
&=\begin{bmatrix}
	\begin{bmatrix}
		\left(W_N^{n\times m}\right)^2
	\end{bmatrix}&
\begin{bmatrix}
	W_N^n\left(W_N^{n\times m}\right)^2
\end{bmatrix}\\
	\begin{bmatrix}
		\left(W_N^{n\times m}\right)^2
	\end{bmatrix}&
\begin{bmatrix}
	-W_N^n\left(W_N^{n\times m}\right)^2
\end{bmatrix}
\end{bmatrix}\\
&=\begin{bmatrix}
	\begin{bmatrix}
		W_{N/2}^{n\times m}
	\end{bmatrix}&
\begin{bmatrix}
	W_{N}^nW_{N/2}^{n\times m}
\end{bmatrix}\\
	\begin{bmatrix}
		W_{N/2}^{n\times m}
	\end{bmatrix}&
\begin{bmatrix}
	-W_N^nW_{N/2}^{n\times m}
\end{bmatrix}
\end{bmatrix}\\
&=\begin{bmatrix}
	\vec{F}_{N/2}&\vec{D}_{N/2}\vec{F}_{N/2}\\
	\vec{F}_{N/2}&-\vec{D}_{N/2}\vec{F}_{N/2}
\end{bmatrix}
\end{align}
	\item Find 
	\begin{align}
		\vec{P}_4 \vec{x}
	\end{align}
\solution Since $\vec{P_4}$ is $4\times4$ matrix and $\vec{x}$ is $6\times1$, so for making them compatible for multiplication we have to remove last 2 component of $\vec{x}$
% pad the $\vec{P_4}$ matrix
%On padding, we get $\vec{P_4}$ matrix as 
%\begin{align}
%	\vec{P}=\begin{bmatrix}
%		1&0&0&0&0&0\\
%		0&0&1&0&0&0\\
%		0&1&0&0&0&0\\
%		0&0&0&1&0&0\\
%	\end{bmatrix}
%\end{align}
%and we know
%\begin{align}
%	\vec{x}=\myvec{
%		1\\
%		2\\
%		3\\
%		4\\
%		2\\
%		1
%	}
%\end{align}
So, we get $\vec{x}$ as 
\begin{align}
	\vec{x}=\myvec{
		1\\
		2\\
		3\\
		4
	}
\end{align}
\begin{align}
	\vec{P}_4\vec{x}&=\vec{P}\vec{x}\\
	&=\begin{bmatrix}
					1&0&0&0\\
					0&0&1&0\\
					0&1&0&0\\
					0&0&0&1\\
	\end{bmatrix}\myvec{
	1\\
	2\\
	3\\
	4
}\\
	&=\myvec{
		1\\
		3\\
		2\\
		4
	}
\end{align}
	\item Show that 
	\begin{align}
		\vec{X} = \vec{F}_N \vec{x}
		\label{eq:dft-mat-def}
	\end{align}
	where $\vec{x}, \vec{X}$ are the vector representations of $x(n), X(k)$ respectively.
	
	\solution Since $\vec{F}_N$ and $\vec{x}$ may not be compatible for matrix product. So for making them compatible either we have to remove some last component of $\vec{x}$ or pad zeros at the end of $\vec{x}$
	
	Let $\vec{x}_1$ be transformed vector.
	
	So,
	\begin{align}
		&\vec{F}_N\vec{x}=\vec{F}_N\vec{x}_1\\
		&=\begin{bmatrix}
			W_N^{0\times0}&W_N^{0\times1}&..&W_N^{0\times N-1}\\
			..&..&..&..\\
			W_N^{N-1 \times 0}&W_N^{N-1 \times 1}&..&W_N^{N-1 \times N-1}
		\end{bmatrix}
	\myvec{x(0)\\x(1)\\..\\x(N-1)}\\
	&=\myvec{\sum_{n=0}^{N-1}W_N^{0\times n} x(n)\\\sum_{n=0}^{N-1}W_N^{1\times n} x(n)\\..\\\sum_{n=0}^{N-1}W_N^{N-1\times n} x(n)}\\
	&=\myvec{\sum_{n=0}^{N-1}e^{\frac{-j2\pi0\times n}{N}} x(n)\\\sum_{n=0}^{N-1}e^{\frac{-j2\pi1\times n}{N}} x(n)\\..\\\sum_{n=0}^{N-1}e^{\frac{-j2\pi (N-1)\times n}{N}} x(n)}\\
	&=\vec{X}
\end{align}
	\item Derive the following Step-by-step visualisation  of
	8-point FFTs into 4-point FFTs and so on
	\begin{equation}
		\begin{bmatrix}
			X(0) \\ 
			X(1) \\ 
			X(2) \\ 
			X(3)
		\end{bmatrix}
		=
		\begin{bmatrix}
			X_{1}(0) \\ 
			X_{1}(1)\\ 
			X_{1}(2)\\
			X_{1}(3)\\
		\end{bmatrix}
		+
		\begin{bmatrix}
			W^{0}_{8} & 0 & 0 & 0\\
			0 & W^{1}_{8} & 0 & 0\\
			0 & 0 & W^{2}_{8} & 0\\
			0 & 0 & 0 & W^{3}_{8}
		\end{bmatrix}
		\begin{bmatrix}
			X_{2}(0) \\ 
			X_{2}(1) \\ 
			X_{2}(2) \\
			X_{2}(3)
		\end{bmatrix}
	\end{equation}
	\begin{equation}
		\begin{bmatrix}
			X(4) \\ 
			X(5) \\ 
			X(6) \\ 
			X(7)
		\end{bmatrix}
		=
		\begin{bmatrix}
			X_{1}(0) \\ 
			X_{1}(1)\\ 
			X_{1}(2)\\
			X_{1}(3)\\
		\end{bmatrix}
		-
		\begin{bmatrix}
			W^{0}_{8} & 0 & 0 & 0\\
			0 & W^{1}_{8} & 0 & 0\\
			0 & 0 & W^{2}_{8} & 0\\
			0 & 0 & 0 & W^{3}_{8}
		\end{bmatrix}
		\begin{bmatrix}
			X_{2}(0) \\ 
			X_{2}(1) \\ 
			X_{2}(2) \\
			X_{2}(3)
		\end{bmatrix}
	\end{equation}
	4-point FFTs into 2-point FFTs
	\begin{equation}
		\begin{bmatrix}
			X_{1}(0) \\ 
			X_{1}(1)\\ 
		\end{bmatrix}
		=
		\begin{bmatrix}
			X_{3}(0) \\ 
			X_{3}(1)\\ 
		\end{bmatrix}
		+
		\begin{bmatrix}
			W^{0}_{4} & 0\\
			0 & W^{1}_{4}
		\end{bmatrix}
		\begin{bmatrix}
			X_{4}(0) \\ 
			X_{4}(1) \\ 
		\end{bmatrix}
	\end{equation}
	\begin{equation}
		\begin{bmatrix}
			X_{1}(2) \\ 
			X_{1}(3)\\ 
		\end{bmatrix}
		=
		\begin{bmatrix}
			X_{3}(0) \\ 
			X_{3}(1)\\ 
		\end{bmatrix}
		-
		\begin{bmatrix}
			W^{0}_{4} & 0\\
			0 & W^{1}_{4}
		\end{bmatrix}
		\begin{bmatrix}
			X_{4}(0) \\ 
			X_{4}(1) \\ 
		\end{bmatrix}
	\end{equation}
	\begin{equation}
		\begin{bmatrix}
			X_{2}(0) \\ 
			X_{2}(1)\\ 
		\end{bmatrix}
		=
		\begin{bmatrix}
			X_{5}(0) \\ 
			X_{5}(1)\\ 
		\end{bmatrix}
		+
		\begin{bmatrix}
			W^{0}_{4} & 0\\
			0 & W^{1}_{4}
		\end{bmatrix}
		\begin{bmatrix}
			X_{6}(0) \\ 
			X_{6}(1) \\ 
		\end{bmatrix}
	\end{equation}
	\begin{equation}
		\begin{bmatrix}
			X_{2}(2) \\ 
			X_{2}(3)\\ 
		\end{bmatrix}
		=
		\begin{bmatrix}
			X_{5}(0) \\ 
			X_{5}(1)\\ 
		\end{bmatrix}
		-
		\begin{bmatrix}
			W^{0}_{4} & 0\\
			0 & W^{1}_{4}
		\end{bmatrix}
		\begin{bmatrix}
			X_{6}(0) \\ 
			X_{6}(1) \\ 
		\end{bmatrix}
	\end{equation}
	\begin{equation}
		P_{8}
		\begin{bmatrix}
			x(0) \\ 
			x(1) \\ 
			x(2) \\ 
			x(3) \\ 
			x(4) \\ 
			x(5) \\
			x(6) \\
			x(7)
		\end{bmatrix}
		= 
		\begin{bmatrix}
			x(0) \\ 
			x(2) \\ 
			x(4) \\ 
			x(6) \\
			x(1) \\ 
			x(3) \\ 
			x(5) \\
			x(7)
		\end{bmatrix}
	\end{equation}
	\begin{equation}
		P_{4}
		\begin{bmatrix}
			x(0) \\ 
			x(2) \\ 
			x(4) \\ 
			x(6) \\
		\end{bmatrix}
		= 
		\begin{bmatrix}
			x(0) \\ 
			x(4) \\ 
			x(2) \\
			x(6)
		\end{bmatrix}
	\end{equation}
	\begin{equation}
		P_{4}
		\begin{bmatrix}
			x(1) \\ 
			x(3) \\ 
			x(5) \\
			x(7)
		\end{bmatrix}
		= 
		\begin{bmatrix}
			x(1) \\ 
			x(5) \\ 
			x(3) \\ 
			x(7) \\
		\end{bmatrix}
	\end{equation}
	Therefore,
	\begin{equation}
		\begin{bmatrix}
			X_{3}(0) \\ 
			X_{3}(1)\\ 
		\end{bmatrix}
		= F_{2}
		\begin{bmatrix}
			x(0) \\ 
			x(4) \\ 
		\end{bmatrix}
	\end{equation}
	\begin{equation}
		\begin{bmatrix}
			X_{4}(0) \\ 
			X_{4}(1)\\ 
		\end{bmatrix}
		= F_{2}
		\begin{bmatrix}
			x(2) \\ 
			x(6) \\ 
		\end{bmatrix}
	\end{equation}
	\begin{equation}
		\begin{bmatrix}
			X_{5}(0) \\ 
			X_{5}(1)\\ 
		\end{bmatrix}
		= F_{2}
		\begin{bmatrix}
			x(1) \\ 
			x(5) \\ 
		\end{bmatrix}
	\end{equation}
	\begin{equation}
		\begin{bmatrix}
			X_{6}(0) \\ 
			X_{6}(1)\\ 
		\end{bmatrix}
		= F_{2}
		\begin{bmatrix}
			x(3) \\ 
			x(7) \\ 
		\end{bmatrix}
	\end{equation}
	\item For 
	\begin{align}
		\vec{x} = \myvec{1\\2\\3\\4\\2\\1}
		\label{eq:equation1}
	\end{align}
	compte the DFT  
	using 
	\eqref{eq:dft-mat-def}
	
	\solution Let N=6. So,
\begin{align}
	&\vec{X}=\vec{F}_6\vec{x}\\	
	&=\begin{bmatrix}
		W_6^{mn}
	\end{bmatrix}\vec{x}\\
&=\begin{bmatrix}
	W_6^0&W_6^0&W_6^0&W_6^0&W_6^0&W_6^0\\
	W_6^0&W_6^1&W_6^2&W_6^3&W_6^4&W_6^5\\
	W_6^0&W_6^2&W_6^4&W_6^6&W_6^8&W_6^{10}\\
	W_6^0&W_6^3&W_6^6&W_6^9&W_6^{12}&W_6^{15}\\
	W_6^0&W_6^4&W_6^8&W_6^{12}&W_6^{16}&W_6^{20}\\
	W_6^0&W_6^5&W_6^{10}&W_6^{15}&W_6^{20}&W_6^{25}\\
\end{bmatrix}\vec{x}\\
&=\begin{bmatrix}
	1&1&1&1&1&1\\
	1&e^{-j\frac{\pi}{3}}&e^{-j\frac{2\pi}{3}}&e^{-j\pi}&e^{-j\frac{4\pi}{3}}&e^{\frac{5\pi}{3}}\\
	1&e^{-j\frac{2\pi}{3}}&e^{-j\frac{4\pi}{3}}&e^{-j2\pi}&e^{-j\frac{8\pi}{3}}&e^{\frac{10\pi}{3}}\\
	1&e^{-j\pi}&e^{-j2\pi}&e^{-j3\pi}&e^{-j4\pi}&e^{-j5\pi}\\
	1&e^{-j\frac{4\pi}{3}}&e^{-j\frac{8\pi}{3}}&e^{-j4\pi}&e^{-j\frac{16\pi}{3}}&e^{\frac{20\pi}{3}}\\
	1&e^{-j\frac{5\pi}{3}}&e^{-j\frac{10\pi}{3}}&e^{-j5\pi}&e^{-j\frac{20\pi}{3}}&e^{\frac{25\pi}{3}}\\
\end{bmatrix}\myvec{1\\2\\3\\4\\2\\1}\\
&=\myvec{13+0j\\
	-4-1.732j\\
	1+0j\\
	-1+0j\\
	1+0j\\
	-4+1.732j}
\end{align}	
	Download and run the following python program.
	\begin{lstlisting}
wget https://github.com/himanshukumargupta11012/EE3900_assignments/blob/master/assignment_1/ques_7/dft_Fx.py
	\end{lstlisting}
	\item Repeat the above exercise using the FFT
	after zero padding $\vec{x}$.
	
	\solution  We know that FFT works for N  which is of the form $2^n$ where $n\in \cal N$.
	So, we have to pad $\vec{x}$ with zeros to its nearest $2^n$ length.
	So,
	\begin{align}
		\vec{x}=\myvec{1\\2\\3\\4\\2\\1\\0\\0}
	\end{align}
We know that if N is even then
\begin{align}
	\vec{F}_N&=\begin{bmatrix}
		\vec{F}_{N/2}&\vec{D}_{N/2}\vec{F}_{N/2}\\
		\vec{F}_{N/2}&-\vec{D}_{N/2}\vec{F}_{N/2}
	\end{bmatrix}\vec{P}_N
\end{align}
So,
\begin{align}
	\vec{F}_2&=\begin{bmatrix}
		\vec{F}_{1}&\vec{D}_{1}\vec{F}_{1}\\
		\vec{F}_{1}&-\vec{D}_{1}\vec{F}_{1}
	\end{bmatrix}\vec{P}_2\\
&=\begin{bmatrix}
	1& 1\\
	1&-1
\end{bmatrix}
\begin{bmatrix}
1& 0\\
0&1
\end{bmatrix}\\
&=\begin{bmatrix}
	1& 1\\
	1&-1
\end{bmatrix}
\end{align}
Now,
\begin{align}
	\vec{D}_{2}\vec{F}_{2}&=diag\myvec{W_{4}^{0} & W_{4}^{1}}\vec{F}_{2}\\
	&=\begin{bmatrix}
		1& 0\\
		0&-j
	\end{bmatrix}
\begin{bmatrix}
	1& 1\\
	1&-1
\end{bmatrix}\\
&=\begin{bmatrix}
	1& 1\\
	-j&j
\end{bmatrix}
\end{align}
So,
\begin{align}
	\vec{F}_4&=\begin{bmatrix}
		\vec{F}_{2}&\vec{D}_{2}\vec{F}_{2}\\
		\vec{F}_{2}&-\vec{D}_{2}\vec{F}_{2}
	\end{bmatrix}\vec{P}_4\\
&=\begin{bmatrix}
	1& 1&1& 1\\
	1&-1&-j&j\\
	1& 1&-1& -1\\
	1&-1&j&-j
\end{bmatrix}\begin{bmatrix}
1&0&0&0\\
0&0&1&0\\
0&1&0&0\\
0&0&0&1\\
\end{bmatrix}\\
&=\begin{bmatrix}
	1&1&1&1\\
	1&-j&-1&j\\
	1&-1&1&-1\\
	1&j&-1&-j\\
\end{bmatrix}
\end{align}
Now,
\begin{align}
	\vec{D}_{4}\vec{F}_{4}&=diag\myvec{W_{8}^{0} & W_{8}^{1}&W_{8}^{2}&W_{8}^{3}}\vec{F}_{4}\\
	&=
\end{align}

	%	    \eqref{eq:fft-mat-def}
	\item Write a C program to compute the 8-point FFT. 
	
	\solution Download and run the following C program.
	\begin{lstlisting}
wget https://github.com/himanshukumargupta11012/EE3900_assignments/blob/master/assignment_1/ques_7/8pnt_fft.c
	\end{lstlisting}
\end{enumerate}
\section{Exercises}
Answer the following questions by looking at the python code in Problem \ref{prob:output}.
\begin{enumerate}[label=\thesection.\arabic*]
	\item
	The command
	\begin{lstlisting}
output_signal = signal.lfilter(b, a, input_signal)
	\end{lstlisting}
	in Problem \ref{prob:output} is executed through the following difference equation
	\begin{equation}
		\label{eq:iir_filter_gen}
		\sum _{m=0}^{M}a\brak{m}y\brak{n-m}=\sum _{k=0}^{N}b\brak{k}x\brak{n-k}
	\end{equation}
	%
	where the input signal is $x(n)$ and the output signal is $y(n)$ with initial values all 0. Replace
	\textbf{signal.filtfilt} with your own routine and verify.
	%
	\item Repeat all the exercises in the previous sections for the above $a$ and $b$.
	\item What is the sampling frequency of the input signal?
	\\
	\solution
	Sampling frequency(fs)=44.1kHZ.
	\item
	What is type, order and  cutoff-frequency of the above butterworth filter
	\\
	\solution
	The given butterworth filter is low pass with order=2 and cutoff-frequency=4kHz.
	%
	\item
	Modifying the code with different input parameters and to get the best possible output.
	%
\end{enumerate}
\end{document}